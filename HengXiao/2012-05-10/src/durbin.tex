%!TEX TS-program = xelatex
%!TEX encoding = UTF-8 Unicode
\documentclass[11pt]{article}
\usepackage{graphicx}
\usepackage{natbib}
\usepackage{enumerate}
\usepackage{fullpage}

\usepackage{amssymb}
\usepackage{amsmath}
\usepackage{algorithm2e}

\usepackage[usenames]{color}
\usepackage{subfig}
\usepackage{fixltx2e}

\newcommand{\rans}[1]{\langle #1 \rangle}
\newcommand{\avg}[1]{\langle #1 \rangle^{\textrm{\tiny AVG}}}
\newcommand{\ke}{$k$--$\varepsilon$ }
\newcommand{\kw}{$k$--$\omega$ }
\newcommand{\uptiny}[1]{\textsuperscript{\tiny (#1)}}

\setlength{\parskip}{\baselineskip}
\setlength{\parindent}{0pt}
\renewcommand{\baselinestretch}{1.35}

\title{Reynolds Stress Model with Elliptic Relaxation}

\author{Heng Xiao}

\begin{document}
\maketitle

\section{Reynolds Averaged Navier Stokes Equations}

General RANS equations:

\begin{subequations} \label{eq:common}
  \begin{align}
  \frac{\partial  U_i}{\partial t}+\frac{\partial
    \left(U_i U_j \right)}{\partial  x_j}  = & 
  - \frac{1}{\rho} \frac{\partial p}{\partial
    x_i}  +\nu\frac{\partial^2 U_i}{\partial x_j \partial
    x_j}   - \frac{\partial \tau_{ij}}{\partial x_j} 
     \label{eq:common-u}  \\
\textrm{and} \quad
\frac{1}{\rho} \frac{\partial^2  p}{\partial x_i\partial x_i}  = &
  - \frac{\partial^2}{\partial x_i\partial
    x_j}\left( U_i  U_j + \tau_{ij}
  \right), \label{eq:common-p}
\end{align}
\end{subequations}
where $t$ and $x_i$ are time and space coordinates, respectively;
$\nu$ is the kinematic viscosity, $\rho$ is the constant fluid
density, and $p$ is the pressure.  $U_i$, $p$, and $\tau_{ij}$
represent Reynolds-averaged velocity $\rans{U_i}$, Reynolds-averaged
pressure $\rans{p}$, and Reynolds stresses $\tau_{ij}$, respectively.

\section{Reynolds Stress Transport Model}

In the Reynolds stress transport model by Durbin (1993), the following
 equations for the Reynolds stress and rate of dissipation are solved:
\begin{align}
\frac{\partial \tau_{ij}}{\partial t}+U_{k}
\frac{\partial  \tau_{ij}}{\partial
  x_{k}} & =
P_{ij}  - \varepsilon_{ij} + D_{ij} + \mathcal{R}_{ij}  \\
\frac{\partial \varepsilon}{\partial t} +  U_k \frac{\partial
  \varepsilon }{\partial x_k} & =  
\frac{C_{\varepsilon_1}^* P - C_{\varepsilon_2} \varepsilon}{T} +
\frac{\partial}{\partial x_{i}}  \left[
  \left( C_{\varepsilon} T \tau_{ij} + \nu \delta_{ij} \right)  \frac{\partial
      \varepsilon} {\partial x_{j}} \right] \label{eqn:epsilon}
\end{align}
where $U_{k}$ is the Reynolds averaged velocity; $\tau_{ij}$ is the
Reynolds stress tensor; and $\varepsilon = \frac{1}{2}
\varepsilon_{ii}$ is the rate of dissipation of turbulent kinetic
energy; $\delta_{ij}$ is the Kronecker delta.

The turbulence stress production term
\begin{equation}
  \label{eq:Pij}
 P_{ij} = -\tau_{ik} \frac{\partial U_{j}}{\partial x_{k}} -  
 \tau_{jk} \frac{\partial U_{i}}{\partial x_{k}}
\end{equation}
is in closed form, and $P = \frac{1}{2}P_{ij}$.  The rate of dissipation
$\varepsilon_{ij}$ and the diffusion $D_{ij}$ of the turbulent
stresses are modeled as:
\begin{align}
\varepsilon_{ij} & = \frac{\varepsilon \tau_{ij}}{k} \\
D_{ij} &  =  \frac{\partial} {\partial x_{k}}
\left( \frac{C_{\mu}T}{\sigma^{(k)}} \tau_{km}
  \frac{\partial \tau_{ij}}{\partial x_{m}} 
\right)
  +  \nu \frac{\partial^2 \tau_{ij}}{ \partial x_k}
\label{eqn:Dij}
\end{align}
with the time scale $T$ defined by:
\begin{equation}
T=\max\left(\frac{k}{\varepsilon}, \;
  C_T\left(\frac{\nu}{\varepsilon}\right)^{\frac{1}{2}}\right).   
\end{equation}

The pressure-rate-of-strain is obtained from:
\begin{subequations}
\begin{align}
\mathcal{R}_{ij}&=k f_{ij},  \\
 L^2\nabla^2f_{ij}-f_{ij}&=-\Pi_{ij}. \label{aeq:f}
\end{align}
\end{subequations}
where the length scale $L$ is formulated as
\begin{equation}
L=C_L \max\left(\frac{k^\frac{3}{2}}{\varepsilon}, \;
  C_{\eta}\left(\frac{\nu^3}{\varepsilon}\right)^{\frac{1}{4}}\right). 
\end{equation}
The source term $\Pi_{ij}$ in Equation~(\ref{aeq:f}) is given by
Rotta's return to isotropy and isotropization of production terms:
\begin{equation}
 \Pi_{ij} =  \frac{1- C_1}{kT} \left( \tau_{ij} -\frac{2}{3}k
 \delta_{ij}  \right) + \frac{C_2}{k} \left( P_{ij} - \frac{2}{3}P \delta_{ij} \right)
\end{equation}

Compared to the standard $\varepsilon$ models, the coefficient
$C_{\varepsilon_1}^*$ in Equation~(\ref{eqn:epsilon}) is modified as
follows $C_{\varepsilon_1}^* = C_{\varepsilon_1} (1 + a_1
P/\varepsilon)$ to account for anisotropic production terms the wall.
The value for all other coefficients in the models above are shown
in Table~\ref{table1}.

\begin{table}[h!]
  \begin{center}
  \caption{\label{table1}Model constants used in the Reynolds stress
    model.} 
  \begin{tabular}{ccccccccccc}
    \hline
    $C_1$ & $C_2$ & $C_{\varepsilon_1}$ &
    $C_{\varepsilon_2}$ & $C_{\eta}$ & $C_{L}$ &
    $C_{\mu}$ & $C_{T}$ & $\sigma^{(\varepsilon)}$ &
    $\sigma^{(k)}$ & $a_1$ \\ 
    \hline
    1.22 & 0.6 & 1.44 & 1.9 & 80 & 0.25 & 0.23 & 6 & 1.65 & 1.2 & 0.1 \\ \hline
  \end{tabular}
\end{center}
\end{table}


The wall boundary conditions for $f_{ij}$ are imposed according to
Manceau and Hanjali\'c (2002) as follows:
\begin{align}
 f_{ij}^{w}&=-\frac{20\nu^2}{\varepsilon}\frac{\tau_{ij}}{y^4}
   \text{ for $f_{22}^{w}$, $f_{12}^{w}$ and 
   $f_{23}^{w}$,} \nonumber \\ 
 f_{ij}^{w}&=-\frac{1}{2}f_{22}^{w} \text{ for $f_{11}^{w}$,
   $f_{33}^{w}$,} \label{eq:fbc} \\
   f_{13}^{w}&=0, \nonumber
\end{align}
where $y$ is the wall distance of the first cell; indexes 1, 2, and 3
indicate streamwise, wall-normal, and spanwise directions,
respectively, based on local coordinate aligned with the wall.  The
boundary conditions for $\tau_{ij}$ and $\varepsilon$ are $\tau_{ij}
= 0$ and $\varepsilon=2\nu k/y^2$.

\section{Implementation Notes}

The following items were observed:
\begin{enumerate}
\item Imposing wall boundary conditions for $f_{ij}$ in a local coordinate
system with $y$ coordinate (or direction 2) aligned with the
wall-normal direction, and the $x$ coordinate aligned with mean flow
direction at the cell nearest to wall. (Equation~(\ref{eq:fbc})).
\item When estimating $y^+ = y/(\nu/\sqrt{\tau_w})$,  a
  ``propagation`` is used, where $y$ is wall-distance, and $\tau_w$ is
  the wall-shear stress.  The functionality is available in OpenFOAM
  as class \verb+wallDistData+, which is used in van Driest damping.
\end{enumerate}

The following measures are taken during the implementation to ensure
stability:
\begin{enumerate}
\item Removed off-diagonal terms for the diffusion coefficient
  $C_{\mu} T \tau_{ij}$. This effectively prevents inverse
  diffusion. (Equation~(\ref{eqn:Dij}))
\item Harmonic interpolation of diffusion coefficient for $D_{ij}$
  (effects not fully validated).
\end{enumerate}


\end{document}  
